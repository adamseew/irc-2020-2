
%%
%% Conference Paper for IRC'20, November 9-11, Taichung, Taiwan
%% ***
%%

\documentclass[conference]{IEEEtran}

\newcommand{\stt}[1]{{\small\tt #1}} %\small\tt too small here
\newcommand{\powprof}{\stt{powprofiler}}
\newcommand{\figpath}{./figures}
\let\labelindent\relax

\usepackage[inline]{enumitem}
\usepackage{booktabs}
\usepackage{flushend}
\usepackage{tikz}

%% citation packege
\usepackage{cite}

%% figures package
%\usepackage[pdftex]{graphicx}
%\graphicspath{{figures/}}
%\DeclareGraphicsExtensions{.pdf,.jpeg,.png}

%% math package
\usepackage[cmex10]{amsmath}
\usepackage{mathtools}
\usepackage{amssymb}

%% pseudocode package
%\usepackage{algorithmic}

%% packages for alignment
%\usepackage{array}
%\usepackage{mdwmath}
%\usepackage{mdwtab}
%\usepackage{eqparbox}

%% packages for subfigures (eventually)
\usepackage[tight,footnotesize]{subfigure}
%S\usepackage[caption=false]{caption}
%\usepackage[font=footnotesize]{subfig}
%\usepackage[caption=false,font=footnotesize]{subfig}

%% package for urls
\usepackage{url}

%% correct bad hyphenation here
\hyphenation{analysis}

%% references (generates a bib file for bibtex)
\begin{filecontents}{\jobname.bib}
\end{filecontents}

\begin{document}

\title{Energy efficient vision based autonomous tracking and landing of a quadrocopter on a moving platform for an agricultural application}

%% author names and affiliations
\author{
\IEEEauthorblockN{Georgios Zamanakos, Adam Seewald, Henrik Skov Midtiby, and Ulrik Pagh Schultz}
\IEEEauthorblockA{SDU UAS Center, M{\ae}rsk Mc-Kinney M{\o}ller Institute\\
University of Southern Denmark\\
Email: \{*\}@mmmi.sdu.dk
}}

%% make the title area
\maketitle


\begin{abstract}

%\boldmath
abstract\\
abstract\\
abstract\\
abstract\\
abstract\\
abstract\\
abstract\\
abstract\\
abstract\\
abstract\\
abstract\\
abstract\\
abstract\\
abstract

\end{abstract}

% For peer review papers, you can put extra information on the cover
% page as needed:
% \ifCLASSOPTIONpeerreview
% \begin{center} \bfseries EDICS Category: 3-BBND \end{center}
% \fi
%
% For peerreview papers, this IEEEtran command inserts a page break and
% creates the second title. It will be ignored for other modes.
\IEEEpeerreviewmaketitle

%%%%%%%%%%%%%%%%%%%%%%
\section{Introduction}
\label{sec:introduction}
In the last years UAVs, especially multicopter drones, are used for various commercial applications such as monitoring, surveillance, transportation of small payloads and agricultural applications. One of the biggest constraints, is their limited flight range due to the battery or fuel capacity. It is therefore seen that to further extend that range, the UAV will have to land in order to change/charge its battery or refuel. Furthermore implementing energy efficient algorithms onboard of the UAV, especially computer vision ones, will further reduce the energy consuption during flight and increase the overall flight range. Concerning the autonomous landing, it is not energy or time efficient for the UAV to fly back to its base. Instead the UAV should land as close as possible to its current location. Constructing and supporting many landing locations scattered around a wide area is considered as a non cost effective solution. Instead it is suggested that the UAV should land on other bigger ground vehicles that are close to its current location, to recharge and then take-off again to continue its mission. 

The use of GPS signal for autonomous landing is not considered as a safe option. The noise and errors of the GPS signal are not predictable and can be further increased by signal reflections from buildings, multipathing errors, bad weather conditions or even by the position of the satellites on a given hour of the day. Also, GPS signal jammers may be used to hack the drone and guide it to a non-safe location or even crash it. Therefore a vision-based autonomous landing system is considered as a more optimal option. A drone equipped with a camera can extract information from the scene, in order to find the location of the landing site and guide itself accordingly. Such a system is more suitable for guidance and navigation in GPS-denied locations and is more robust towards hacking attempts.

In this project an agricultural case of a quadrocopter tracking and landing on a moving tractor will be studied. Furthermore, the detection and tracking of the moving ground vehicle for inspection purposes will also be studied.



*

%%%%%%%%%%%%%%%%%%%%%%
\section{Related Work}
\label{sec:related}

*

%%%%%%%%%%%%%%%%%%%%%%%%%%%
\section{Approach}
\label{sec:approach}

*

%%%%%%%%%%%%%%%%%%%%
\section{Evaluation}
\label{sec:experimental}

*

%%%%%%%%%%%%%%%%%%%%%%%%%%%%%%%%%%%%
\section{Conclusion and Future Work}
\label{sec:conclusion}

*

%% acknowledgement
%%%%%%%%%%%%%%%%%%%%%%%%%
\section*{Acknowledgment}

This work is supported and partly funded by the European Union’s Horizon2020 research and innovation program under grant agreement No. 779882 (TeamPlay).

\bibliographystyle{IEEEtran}
\bibliography{\jobname} 
\vspace{1ex}

\end{document}
